If the data is clustered, one way to handle the clustering is to use a
multilevel modeling approach. In the SEM framework, this leads to
multilevel SEM. The multilevel capabilities of lavaan are still limited,
but you can fit a two-level SEM with random intercepts (note: only when
all data is continuous and complete; listwise deletion is currently used
for cases with missing values).

\hypertarget{multilevel-sem-model-syntax}{%
\paragraph{Multilevel SEM model
syntax}\label{multilevel-sem-model-syntax}}

To fit a two-level SEM, you must specify a model for both levels, as
follows:

\begin{Shaded}
\begin{Highlighting}[]
\NormalTok{    model <-}\StringTok{ '}
\StringTok{        level: 1}
\StringTok{            fw =~ y1 + y2 + y3}
\StringTok{            fw ~ x1 + x2 + x3}
\StringTok{        level: 2}
\StringTok{            fb =~ y1 + y2 + y3}
\StringTok{            fb ~ w1 + w2}
\StringTok{    '}
\end{Highlighting}
\end{Shaded}

This model syntax contains two blocks, one for level 1, and one for
level 2. Within each block, you can specify a model just like in the
single-level case. To fit this model, using a toy dataset
\texttt{Demo.twolevel} that is part of the lavaan package, you need to
add the \texttt{cluster=} argument to the sem/lavaan function call:

\begin{Shaded}
\begin{Highlighting}[]
\NormalTok{    fit <-}\StringTok{ }\KeywordTok{sem}\NormalTok{(}\DataTypeTok{model =}\NormalTok{ model, }\DataTypeTok{data =}\NormalTok{ Demo.twolevel, }\DataTypeTok{cluster =} \StringTok{"cluster"}\NormalTok{)}
\end{Highlighting}
\end{Shaded}

The output looks similar to a multigroup SEM output, but where the two
groups are now the within and the between level respectively.

\begin{Shaded}
\begin{Highlighting}[]
    \KeywordTok{summary}\NormalTok{(fit)}
\end{Highlighting}
\end{Shaded}

\begin{verbatim}
## lavaan (0.6-1) converged normally after  36 iterations
## 
##   Number of observations                          2500
##   Number of clusters [cluster]                     200
## 
##   Estimator                                         ML
##   Model Fit Test Statistic                       8.092
##   Degrees of freedom                                10
##   P-value (Chi-square)                           0.620
## 
## Parameter Estimates:
## 
##   Information                                 Observed
##   Observed information based on                Hessian
##   Standard Errors                             Standard
## 
## 
## Level 1 [within]:
## 
## Latent Variables:
##                    Estimate  Std.Err  z-value  P(>|z|)
##   fw =~                                               
##     y1                1.000                           
##     y2                0.774    0.034   22.671    0.000
##     y3                0.734    0.033   22.355    0.000
## 
## Regressions:
##                    Estimate  Std.Err  z-value  P(>|z|)
##   fw ~                                                
##     x1                0.510    0.023   22.037    0.000
##     x2                0.407    0.022   18.273    0.000
##     x3                0.205    0.021    9.740    0.000
## 
## Intercepts:
##                    Estimate  Std.Err  z-value  P(>|z|)
##    .y1                0.000                           
##    .y2                0.000                           
##    .y3                0.000                           
##    .fw                0.000                           
## 
## Variances:
##                    Estimate  Std.Err  z-value  P(>|z|)
##    .y1                0.986    0.046   21.591    0.000
##    .y2                1.066    0.039   27.271    0.000
##    .y3                1.011    0.037   27.662    0.000
##    .fw                0.546    0.040   13.539    0.000
## 
## 
## Level 2 [cluster]:
## 
## Latent Variables:
##                    Estimate  Std.Err  z-value  P(>|z|)
##   fb =~                                               
##     y1                1.000                           
##     y2                0.717    0.052   13.824    0.000
##     y3                0.587    0.048   12.329    0.000
## 
## Regressions:
##                    Estimate  Std.Err  z-value  P(>|z|)
##   fb ~                                                
##     w1                0.165    0.079    2.093    0.036
##     w2                0.131    0.076    1.715    0.086
## 
## Intercepts:
##                    Estimate  Std.Err  z-value  P(>|z|)
##    .y1                0.024    0.075    0.327    0.743
##    .y2               -0.016    0.060   -0.269    0.788
##    .y3               -0.042    0.054   -0.777    0.437
##    .fb                0.000                           
## 
## Variances:
##                    Estimate  Std.Err  z-value  P(>|z|)
##    .y1                0.058    0.047    1.213    0.225
##    .y2                0.120    0.031    3.825    0.000
##    .y3                0.149    0.028    5.319    0.000
##    .fb                0.899    0.118    7.592    0.000
\end{verbatim}

After fitting the model, you can inspect the intra-class correlations:

\begin{Shaded}
\begin{Highlighting}[]
    \KeywordTok{lavInspect}\NormalTok{(fit, }\StringTok{"icc"}\NormalTok{)}
\end{Highlighting}
\end{Shaded}

\begin{verbatim}
##    y1    y2    y3    x1    x2    x3 
## 0.331 0.263 0.232 0.000 0.000 0.000
\end{verbatim}

The see the unrestricted (h1) within and between means and covariances,
you can use

\begin{Shaded}
\begin{Highlighting}[]
    \KeywordTok{lavInspect}\NormalTok{(fit, }\StringTok{"h1"}\NormalTok{)}
\end{Highlighting}
\end{Shaded}

\begin{verbatim}
## $within
## $within$cov
##    y1     y2     y3     x1     x2     x3    
## y1  2.000                                   
## y2  0.789  1.674                            
## y3  0.749  0.564  1.557                     
## x1  0.489  0.393  0.376  0.982              
## x2  0.416  0.322  0.299  0.001  1.011       
## x3  0.221  0.160  0.155 -0.006  0.008  1.045
## 
## $within$mean
##     y1     y2     y3     x1     x2     x3 
##  0.001 -0.002 -0.001 -0.007 -0.003  0.020 
## 
## 
## $cluster
## $cluster$cov
##    y1     y2     y3     w1     w2    
## y1  0.992                            
## y2  0.668  0.598                     
## y3  0.548  0.391  0.469              
## w1  0.125  0.119  0.036  0.870       
## w2  0.086  0.057  0.130 -0.128  0.931
## 
## $cluster$mean
##     y1     y2     y3     w1     w2 
##  0.019 -0.017 -0.043  0.052 -0.091
\end{verbatim}

\hypertarget{important-notes}{%
\paragraph{Important notes}\label{important-notes}}

\begin{itemize}
\item
  note that in \texttt{level:\ 1} the colon follows the \texttt{level}
  keyword; if you type \texttt{level\ 1:}, you will get an error
\item
  you must specify a model for each level; the following syntax is not
  allowed and will produce an error:

\begin{Shaded}
\begin{Highlighting}[]
\NormalTok{    model <-}\StringTok{ '}
\StringTok{        level: 1}
\StringTok{            fw =~ y1 + y2 + y3}
\StringTok{            fw ~ x1 + x2 + x3}
\StringTok{        level: 2}
\StringTok{    '}
\end{Highlighting}
\end{Shaded}
\item
  if you do not have a model in mind for level 2, you can specify a
  saturated level by adding all variances and covariances of the
  endogenous variables (here: y1, y2 and y3):

\begin{Shaded}
\begin{Highlighting}[]
\NormalTok{    model <-}\StringTok{ '}
\StringTok{        level: 1}
\StringTok{            fw =~ y1 + y2 + y3}
\StringTok{            fw ~ x1 + x2 + x3}
\StringTok{        level: 2}
\StringTok{            y1 ~~ y1 + y2 + y3}
\StringTok{            y2 ~~ y2 + y3}
\StringTok{            y3 ~~ y3}
\StringTok{    '}
\end{Highlighting}
\end{Shaded}
\end{itemize}

\hypertarget{convergence-issues-and-solutions}{%
\paragraph{Convergence issues and
solutions}\label{convergence-issues-and-solutions}}

By default, the current version of lavaan (0.6) uses a quasi-Newton
procedure to maximize the loglikelihood of the data given the model
(just like in the single-level case). For most model and data
combinations, this will work fine (and fast). However, every now and
then, you may experience convergence issues.

Non-convergence is typically a sign that something is not quite right
with either your model, or your data. Typical settings are: a small
number of clusters, in combination with (almost) no variance of an
endogenous variable at the between level.

However, if you believe nothing is wrong, you may want to try another
optimization procedure. The current version of lavaan allows for using
the Expectation Maximization (EM) algorithm as an alternative. To switch
to the EM algorithm, you can use:

\begin{Shaded}
\begin{Highlighting}[]
\NormalTok{    fit <-}\StringTok{ }\KeywordTok{sem}\NormalTok{(}\DataTypeTok{model =}\NormalTok{ model, }\DataTypeTok{data =}\NormalTok{ Demo.twolevel, }\DataTypeTok{cluster =} \StringTok{"cluster"}\NormalTok{,}
               \DataTypeTok{verbose =} \OtherTok{TRUE}\NormalTok{, }\DataTypeTok{optim.method =} \StringTok{"em"}\NormalTok{)}
\end{Highlighting}
\end{Shaded}

As the EM algorithm is not accelerated yet, this may take a long time.
It is not unusual that more than 10000 iterations are needed to reach a
solution. To control when the EM algorithm stops, you can set the
stopping criteria as follows:

\begin{Shaded}
\begin{Highlighting}[]
\NormalTok{    fit <-}\StringTok{ }\KeywordTok{sem}\NormalTok{(}\DataTypeTok{model =}\NormalTok{ model, }\DataTypeTok{data =}\NormalTok{ Demo.twolevel, }\DataTypeTok{cluster =} \StringTok{"cluster"}\NormalTok{,}
               \DataTypeTok{verbose =} \OtherTok{TRUE}\NormalTok{, }\DataTypeTok{optim.method =} \StringTok{"em"}\NormalTok{, }\DataTypeTok{em.iter.max =} \DecValTok{20000}\NormalTok{,}
               \DataTypeTok{em.fx.tol =} \FloatTok{1e-08}\NormalTok{, }\DataTypeTok{em.dx.tol =} \FloatTok{1e-04}\NormalTok{)}
\end{Highlighting}
\end{Shaded}

The \texttt{em.fx.tol} argument is used to monitor the change in
loglikelihood between the current step and the previous step. If this
change is smaller than \texttt{em.fx.tol}, the algorithm stops. The
\texttt{em.dx.tol} argument is used to monitor the (unscaled) gradient.
When a solution is reached, all elements of the gradient should be near
zero. When the largest gradient element is smaller than
\texttt{em.dx.tol}, the algorithm stops.

A word of caution: the EM algorithm can always be forced to `converge'
(perhaps after changing the stopping criteria), but that does not mean
you have a model/dataset combination that deserves to converge.
