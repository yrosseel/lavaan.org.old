By and large, structural equation models are used to model the
covariance matrix of the observed variables in a dataset. But in some
applications, it is useful to bring in the means of the observed
variables too. One way to do this is to explicitly refer to intercepts
in the lavaan syntax. This can be done by including `intercept formulas'
in the model syntax. An intercept formula has the following form:

\begin{verbatim}
variable ~ 1
\end{verbatim}

The left part of the expression contains the name of the observed or
latent variable. The right part contains the number \texttt{1},
representing the intercept. For example, in the three-factor H\&S CFA
model, we can add the intercepts of the observed variables as follows:

\begin{verbatim}
# three-factor model
  visual  =~ x1 + x2 + x3
  textual =~ x4 + x5 + x6
  speed   =~ x7 + x8 + x9
# intercepts
  x1 ~ 1
  x2 ~ 1
  x3 ~ 1
  x4 ~ 1
  x5 ~ 1
  x6 ~ 1
  x7 ~ 1
  x8 ~ 1
  x9 ~ 1
\end{verbatim}

However, it is more convenient to omit the intercept formulas in the
model syntax (unless you want to fix their values), and to add the
argument \texttt{meanstructure\ =\ TRUE} in the fitting function. For
example, we can refit the three-factor H\&S CFA model as follows:

\begin{Shaded}
\begin{Highlighting}[]
\NormalTok{fit }\OtherTok{\textless{}{-}} \FunctionTok{cfa}\NormalTok{(HS.model, }
           \AttributeTok{data =}\NormalTok{ HolzingerSwineford1939, }
           \AttributeTok{meanstructure =} \ConstantTok{TRUE}\NormalTok{)}
\FunctionTok{summary}\NormalTok{(fit) }
\end{Highlighting}
\end{Shaded}

\begin{verbatim}
lavaan 0.6-11 ended normally after 35 iterations

  Estimator                                         ML
  Optimization method                           NLMINB
  Number of model parameters                        30
                                                      
  Number of observations                           301
                                                      
Model Test User Model:
                                                      
  Test statistic                                85.306
  Degrees of freedom                                24
  P-value (Chi-square)                           0.000

Parameter Estimates:

  Standard errors                             Standard
  Information                                 Expected
  Information saturated (h1) model          Structured

Latent Variables:
                   Estimate  Std.Err  z-value  P(>|z|)
  visual =~                                           
    x1                1.000                           
    x2                0.554    0.100    5.554    0.000
    x3                0.729    0.109    6.685    0.000
  textual =~                                          
    x4                1.000                           
    x5                1.113    0.065   17.014    0.000
    x6                0.926    0.055   16.703    0.000
  speed =~                                            
    x7                1.000                           
    x8                1.180    0.165    7.152    0.000
    x9                1.082    0.151    7.155    0.000

Covariances:
                   Estimate  Std.Err  z-value  P(>|z|)
  visual ~~                                           
    textual           0.408    0.074    5.552    0.000
    speed             0.262    0.056    4.660    0.000
  textual ~~                                          
    speed             0.173    0.049    3.518    0.000

Intercepts:
                   Estimate  Std.Err  z-value  P(>|z|)
   .x1                4.936    0.067   73.473    0.000
   .x2                6.088    0.068   89.855    0.000
   .x3                2.250    0.065   34.579    0.000
   .x4                3.061    0.067   45.694    0.000
   .x5                4.341    0.074   58.452    0.000
   .x6                2.186    0.063   34.667    0.000
   .x7                4.186    0.063   66.766    0.000
   .x8                5.527    0.058   94.854    0.000
   .x9                5.374    0.058   92.546    0.000
    visual            0.000                           
    textual           0.000                           
    speed             0.000                           

Variances:
                   Estimate  Std.Err  z-value  P(>|z|)
   .x1                0.549    0.114    4.833    0.000
   .x2                1.134    0.102   11.146    0.000
   .x3                0.844    0.091    9.317    0.000
   .x4                0.371    0.048    7.779    0.000
   .x5                0.446    0.058    7.642    0.000
   .x6                0.356    0.043    8.277    0.000
   .x7                0.799    0.081    9.823    0.000
   .x8                0.488    0.074    6.573    0.000
   .x9                0.566    0.071    8.003    0.000
    visual            0.809    0.145    5.564    0.000
    textual           0.979    0.112    8.737    0.000
    speed             0.384    0.086    4.451    0.000
\end{verbatim}

As you can see in the output, the model includes intercept parameters
for both the observed and latent variables. By default, the
\texttt{cfa()} and \texttt{sem()} functions fix the latent variable
intercepts (which in this case correspond to the latent \emph{means}) to
zero. Otherwise, the model would not be estimable. Note that the
chi-square statistic and the number of degrees of freedom is the same as
in the original model (without a mean structure). The reason is that we
brought in some new data (a mean value for each of the 9 observed
variables), but we also added 9 additional parameters to the model (an
intercept for each of the 9 observed variables). The end result is an
identical fit. In practice, the only reason why a user would add
intercept-formulas in the model syntax, is because some constraints must
be specified on them. For example, suppose that we wish to fix the
intercepts of the variables \texttt{x1}, \texttt{x2}, \texttt{x3} and
\texttt{x4} to, say, 0.5. We would write the model syntax as follows:

\begin{verbatim}
# three-factor model
  visual  =~ x1 + x2 + x3
  textual =~ x4 + x5 + x6
  speed   =~ x7 + x8 + x9
# intercepts with fixed values
  x1 + x2 + x3 + x4 ~ 0.5*1
\end{verbatim}

where we have used the left-hand side of the formula to `repeat' the
right-hand side for each element of the left-hand side.
