If a measurement model contains multiple latent variables (factors), we
usually know which indicators belong to each factor. We call this the
factor structure. Confirmatory factor analysis can be used to check if
this a priori factor structure holds in the data. There are settings,
however, where the factor structure is unclear, and we wish to rotate
the solution in order to find a suitable structure in a given model.
When the model also includes a structural part (i.e., regressions among
the latent variables), this is referred to as exploratory structural
equation modeling or ESEM. If there is only a measurement part, this is
called exploratory factor analysis (EFA). What they have in common is
that the factor structure (for one or more blocks) is found by means of
rotation.

\hypertarget{esem}{%
\paragraph{ESEM}\label{esem}}

To illustrate how ESEM works in lavaan, consider the following syntax:

\begin{Shaded}
\begin{Highlighting}[]
\NormalTok{model }\OtherTok{\textless{}{-}} \StringTok{\textquotesingle{}}
\StringTok{    \# efa block 1}
\StringTok{    efa("efa1")*f1 + }
\StringTok{    efa("efa1")*f2 =\textasciitilde{} x1 + x2 + x3 + x4 + x5 + x6}

\StringTok{    \# efa block 2}
\StringTok{    efa("efa2")*f3 + }
\StringTok{    efa("efa2")*f4 =\textasciitilde{} y1 + y2 + y3 + y4 + y5 + y6}

\StringTok{    \# cfa block}
\StringTok{    f5 =\textasciitilde{} z7 + z8 + z9}
\StringTok{    f6 =\textasciitilde{} z10 + z11 + z12}

\StringTok{    \# regressions}
\StringTok{    f3 \textasciitilde{} f1 + f2}
\StringTok{    f4 \textasciitilde{} f3}
\StringTok{\textquotesingle{}}
\end{Highlighting}
\end{Shaded}

This model syntax defines six latent variables (or factors). For f5 and
f6, the factor structure is known, and they belong to a regular CFA
block. But for f1 and f2, the factor structure is not known, and we will
use a rotation method to find an appropiate structure. The f1 and f2
factors belong together in an EFA block that is (arbitrarily) named
\texttt{efa1}. The \texttt{efa("efa1")*} modifier just before f1 and f2
is used to alert lavaan that these two factors belong to the same EFA
block. The factors f3 and f4 belong to a different EFA block (named
\texttt{efa2}) and will be rotated independently. The structural part of
the model is given as usual. To fit this model, we could call the
\texttt{sem()} function as follows:

\begin{Shaded}
\begin{Highlighting}[]
\NormalTok{fit }\OtherTok{\textless{}{-}} \FunctionTok{sem}\NormalTok{(}\AttributeTok{model =}\NormalTok{ model, }\AttributeTok{data =}\NormalTok{ myData, }\AttributeTok{rotation =} \StringTok{"geomin"}\NormalTok{)}
\end{Highlighting}
\end{Shaded}

Different rotation criteria are available, and many rotation options can
be provided (see the manual page for the \texttt{efa()} function for an
overview).

To illustrate ESEM, we will borrow an example from the Mplus User's
Guide (example 5.25). First we read in the data:

\begin{Shaded}
\begin{Highlighting}[]
\NormalTok{ex5\_25 }\OtherTok{\textless{}{-}} \FunctionTok{read.table}\NormalTok{(}\StringTok{"http://statmodel.com/usersguide/chap5/ex5.25.dat"}\NormalTok{)}
\FunctionTok{names}\NormalTok{(ex5\_25) }\OtherTok{=} \FunctionTok{paste0}\NormalTok{(}\StringTok{"y"}\NormalTok{,}\DecValTok{1}\SpecialCharTok{:}\DecValTok{12}\NormalTok{)}
\end{Highlighting}
\end{Shaded}

The model syntax contains a single EFA block (\texttt{efa1} for factors
f1 and f2) and single CFA block (for f3 and f4):

\begin{Shaded}
\begin{Highlighting}[]
\NormalTok{model }\OtherTok{\textless{}{-}} \StringTok{\textquotesingle{}}
\StringTok{    \# efa block }
\StringTok{    efa("efa1")*f1 + }
\StringTok{    efa("efa1")*f2 =\textasciitilde{} y1 + y2 + y3 + y4 + y5 + y6}

\StringTok{    \# cfa block}
\StringTok{    f3 =\textasciitilde{} y7 + y8 + y9}
\StringTok{    f4 =\textasciitilde{} y10 + y11 + y12}

\StringTok{    \# regressions}
\StringTok{    f3 \textasciitilde{} f1 + f2}
\StringTok{    f4 \textasciitilde{} f3}
\StringTok{\textquotesingle{}}
\end{Highlighting}
\end{Shaded}

The following command illustrates the use of various rotation arguments:

\begin{Shaded}
\begin{Highlighting}[]
\NormalTok{fit }\OtherTok{\textless{}{-}} \FunctionTok{sem}\NormalTok{(}\AttributeTok{model =}\NormalTok{ model, }\AttributeTok{data =}\NormalTok{ ex5\_25, }\AttributeTok{rotation =} \StringTok{"geomin"}\NormalTok{,}
           \CommentTok{\# mimic Mplus}
           \AttributeTok{information =} \StringTok{"observed"}\NormalTok{,}
           \AttributeTok{rotation.args =} \FunctionTok{list}\NormalTok{(}\AttributeTok{rstarts =} \DecValTok{30}\NormalTok{, }\AttributeTok{row.weights =} \StringTok{"none"}\NormalTok{,}
                                \AttributeTok{algorithm =} \StringTok{"gpa"}\NormalTok{, }\AttributeTok{std.ov =} \ConstantTok{TRUE}\NormalTok{,}
                                \AttributeTok{geomin.epsilon =} \FloatTok{0.0001}\NormalTok{))}
\FunctionTok{summary}\NormalTok{(fit)}
\end{Highlighting}
\end{Shaded}

\begin{verbatim}
lavaan 0.6-13.1762 ended normally after 35 iterations

  Estimator                                         ML
  Optimization method                           NLMINB
  Number of model parameters                        32

  Rotation method                       GEOMIN OBLIQUE
  Geomin epsilon                                 1e-04
  Rotation algorithm (rstarts)                GPA (30)
  Standardized metric                             TRUE
  Row weights                                     None

  Number of observations                           500

Model Test User Model:
                                                      
  Test statistic                                51.353
  Degrees of freedom                                46
  P-value (Chi-square)                           0.272

Parameter Estimates:

  Standard errors                             Standard
  Information                                 Observed
  Observed information based on                Hessian

Latent Variables:
                   Estimate  Std.Err  z-value  P(>|z|)
  f1 =~ efa1                                          
    y1                0.751    0.048   15.621    0.000
    y2                0.858    0.042   20.469    0.000
    y3                0.736    0.045   16.343    0.000
    y4                0.036    0.051    0.712    0.476
    y5               -0.028    0.049   -0.564    0.573
    y6                0.002    0.003    0.694    0.488
  f2 =~ efa1                                          
    y1                0.034    0.045    0.758    0.449
    y2               -0.002    0.015   -0.151    0.880
    y3               -0.008    0.035   -0.219    0.827
    y4                0.763    0.050   15.374    0.000
    y5                0.810    0.048   16.796    0.000
    y6                0.802    0.041   19.467    0.000
  f3 =~                                               
    y7                1.000                           
    y8                0.894    0.021   41.936    0.000
    y9                0.902    0.021   42.479    0.000
  f4 =~                                               
    y10               1.000                           
    y11               0.734    0.028   26.424    0.000
    y12               0.684    0.028   24.405    0.000

Regressions:
                   Estimate  Std.Err  z-value  P(>|z|)
  f3 ~                                                
    f1                0.493    0.058    8.455    0.000
    f2                0.721    0.057   12.755    0.000
  f4 ~                                                
    f3                0.546    0.032   16.975    0.000

Covariances:
                   Estimate  Std.Err  z-value  P(>|z|)
  f1 ~~                                               
    f2                0.479    0.053    9.072    0.000

Variances:
                   Estimate  Std.Err  z-value  P(>|z|)
   .y1                0.376    0.034   11.064    0.000
   .y2                0.290    0.035    8.239    0.000
   .y3                0.406    0.034   11.817    0.000
   .y4                0.408    0.035   11.742    0.000
   .y5                0.329    0.033   10.046    0.000
   .y6                0.393    0.035   11.073    0.000
   .y7                0.183    0.019    9.796    0.000
   .y8                0.191    0.017   11.269    0.000
   .y9                0.181    0.017   10.812    0.000
   .y10               0.240    0.027    8.746    0.000
   .y11               0.183    0.017   10.791    0.000
   .y12               0.213    0.018   11.998    0.000
    f1                1.000                           
    f2                1.000                           
   .f3                0.527    0.049   10.644    0.000
   .f4                0.565    0.049   11.488    0.000
\end{verbatim}

\hypertarget{exploratory-factor-analysis-efa}{%
\paragraph{Exploratory factor analysis
(EFA)}\label{exploratory-factor-analysis-efa}}

When there is no structural part (i.e., no regressions among the latent
variables) and there is only a single EFA block, then ESEM reduces to
exploratory factor analysis (EFA). Using the Holzinger and Swineford
data, we could specify an EFA with three factors as follows:

\begin{Shaded}
\begin{Highlighting}[]
\NormalTok{efa.model }\OtherTok{\textless{}{-}} \StringTok{\textquotesingle{}}
\StringTok{    efa("efa")*f1 + }
\StringTok{    efa("efa")*f2 + }
\StringTok{    efa("efa")*f3 =\textasciitilde{} x1 + x2 + x3 + x4 + x5 + x6 + x7 + x8 + x9}
\StringTok{\textquotesingle{}}
\NormalTok{fit }\OtherTok{\textless{}{-}} \FunctionTok{cfa}\NormalTok{(efa.model, }\AttributeTok{data =}\NormalTok{ HolzingerSwineford1939)}
\FunctionTok{summary}\NormalTok{(fit, }\AttributeTok{standardized =} \ConstantTok{TRUE}\NormalTok{)}
\end{Highlighting}
\end{Shaded}

\begin{verbatim}
lavaan 0.6-13.1762 ended normally after 1 iteration

  Estimator                                         ML
  Optimization method                           NLMINB
  Number of model parameters                        33

  Rotation method                       GEOMIN OBLIQUE
  Geomin epsilon                                 0.001
  Rotation algorithm (rstarts)                GPA (30)
  Standardized metric                             TRUE
  Row weights                                     None

  Number of observations                           301

Model Test User Model:
                                                      
  Test statistic                                22.897
  Degrees of freedom                                12
  P-value (Chi-square)                           0.029

Parameter Estimates:

  Standard errors                             Standard
  Information                                 Expected
  Information saturated (h1) model          Structured

Latent Variables:
                   Estimate  Std.Err  z-value  P(>|z|)   Std.lv  Std.all
  f1 =~ efa                                                             
    x1                0.712    0.092    7.771    0.000    0.712    0.611
    x2                0.628    0.104    6.063    0.000    0.628    0.534
    x3                0.796    0.096    8.255    0.000    0.796    0.705
    x4                0.011    0.011    0.944    0.345    0.011    0.009
    x5               -0.107    0.089   -1.203    0.229   -0.107   -0.083
    x6                0.076    0.073    1.028    0.304    0.076    0.069
    x7               -0.278    0.109   -2.538    0.011   -0.278   -0.255
    x8                0.012    0.008    1.371    0.170    0.012    0.011
    x9                0.314    0.076    4.142    0.000    0.314    0.312
  f2 =~ efa                                                             
    x1                0.198    0.103    1.917    0.055    0.198    0.170
    x2                0.039    0.092    0.424    0.672    0.039    0.033
    x3               -0.106    0.111   -0.963    0.335   -0.106   -0.094
    x4                0.981    0.058   16.850    0.000    0.981    0.844
    x5                1.153    0.074   15.545    0.000    1.153    0.895
    x6                0.886    0.062   14.338    0.000    0.886    0.810
    x7                0.011    0.012    0.923    0.356    0.011    0.010
    x8               -0.075    0.066   -1.135    0.256   -0.075   -0.074
    x9               -0.002    0.007   -0.315    0.753   -0.002   -0.002
  f3 =~ efa                                                             
    x1                0.015    0.048    0.302    0.762    0.015    0.012
    x2               -0.166    0.092   -1.813    0.070   -0.166   -0.141
    x3                0.002    0.048    0.036    0.971    0.002    0.002
    x4                0.004    0.047    0.091    0.927    0.004    0.004
    x5                0.012    0.036    0.322    0.747    0.012    0.009
    x6               -0.017    0.041   -0.409    0.683   -0.017   -0.015
    x7                0.843    0.105    7.999    0.000    0.843    0.775
    x8                0.752    0.076    9.893    0.000    0.752    0.744
    x9                0.484    0.070    6.954    0.000    0.484    0.481

Covariances:
                   Estimate  Std.Err  z-value  P(>|z|)   Std.lv  Std.all
  f1 ~~                                                                 
    f2                0.373    0.118    3.173    0.002    0.373    0.373
    f3                0.432    0.097    4.465    0.000    0.432    0.432
  f2 ~~                                                                 
    f3                0.306    0.081    3.775    0.000    0.306    0.306

Variances:
                   Estimate  Std.Err  z-value  P(>|z|)   Std.lv  Std.all
   .x1                0.696    0.087    8.038    0.000    0.696    0.513
   .x2                1.035    0.102   10.151    0.000    1.035    0.749
   .x3                0.692    0.097    7.134    0.000    0.692    0.543
   .x4                0.377    0.048    7.902    0.000    0.377    0.279
   .x5                0.403    0.061    6.590    0.000    0.403    0.243
   .x6                0.365    0.042    8.613    0.000    0.365    0.305
   .x7                0.594    0.106    5.624    0.000    0.594    0.502
   .x8                0.479    0.080    5.958    0.000    0.479    0.469
   .x9                0.551    0.060    9.132    0.000    0.551    0.543
    f1                1.000                               1.000    1.000
    f2                1.000                               1.000    1.000
    f3                1.000                               1.000    1.000
\end{verbatim}

In version 0.6-13, we added added the \texttt{efa()} function to
simplify the input, and to produce output that is more in line with
traditional EFA software in R. There is no need to create a model
syntax. You only need to provide the data, and the number of factors.
Instead of a single number, you can also specify a range of numbers. For
example:

\begin{Shaded}
\begin{Highlighting}[]
\NormalTok{var.names }\OtherTok{\textless{}{-}} \FunctionTok{paste}\NormalTok{(}\StringTok{"x"}\NormalTok{, }\DecValTok{1}\SpecialCharTok{:}\DecValTok{9}\NormalTok{, }\AttributeTok{sep =} \StringTok{""}\NormalTok{)}
\NormalTok{fit }\OtherTok{\textless{}{-}} \FunctionTok{efa}\NormalTok{(}\AttributeTok{data =}\NormalTok{ HolzingerSwineford1939[,var.names], }\AttributeTok{nfactors =} \DecValTok{1}\SpecialCharTok{:}\DecValTok{3}\NormalTok{)}
\FunctionTok{summary}\NormalTok{(fit)}
\end{Highlighting}
\end{Shaded}

\begin{verbatim}
This is lavaan 0.6-13.1762 -- running exploratory factor analysis

  Estimator                                         ML
  Rotation method                       GEOMIN OBLIQUE
  Geomin epsilon                                 0.001
  Rotation algorithm (rstarts)                GPA (30)
  Standardized metric                             TRUE
  Row weights                                     None

  Number of observations                           301

Overview models:
                 chisq df pvalue rmsea
  nfactors = 1 312.264 27  0.000 0.187
  nfactors = 2 130.306 19  0.000 0.140
  nfactors = 3  22.897 12  0.029 0.055

Eigenvalues correlation matrix:

    ev1     ev2     ev3     ev4     ev5     ev6     ev7     ev8     ev9 
  3.216   1.639   1.365   0.699   0.584   0.500   0.473   0.286   0.238 

Number of factors:  1 

Standardized loadings: (* = significant at 1% level)

       f1       unique.var   communalities
x1  0.438*           0.808           0.192
x2      .*           0.951           0.049
x3      .*           0.950           0.050
x4  0.848*           0.281           0.719
x5  0.841*           0.293           0.707
x6  0.838*           0.298           0.702
x7      .*           0.967           0.033
x8      .*           0.960           0.040
x9  0.307*           0.906           0.094

                           f1
Sum of squared loadings 2.586
Proportion var          0.287
Cumulative var          0.287
Proportion of total     1.000

Number of factors:  2 

Standardized loadings: (* = significant at 1% level)

       f1      f2       unique.var   communalities
x1      .*  0.430*           0.673           0.327
x2      .       .*           0.906           0.094
x3          0.456*           0.783           0.217
x4  0.851*                   0.274           0.726
x5  0.868*                   0.264           0.736
x6  0.825*                   0.302           0.698
x7          0.448*           0.802           0.198
x8          0.627*           0.630           0.370
x9          0.734*           0.458           0.542

                              f1    f2 total
Sum of sq (obliq) loadings 2.280 1.629 3.909
Proportion var             0.253 0.181 0.434
Cumulative var             0.253 0.434 0.434
Proportion of total        0.583 0.417 1.000

Factor correlations:

      f1    f2
f1 1.000      
f2 0.339 1.000

Number of factors:  3 

Standardized loadings: (* = significant at 1% level)

       f1      f2      f3       unique.var   communalities
x1  0.611*      .                    0.513           0.487
x2  0.534*              .            0.749           0.251
x3  0.705*                           0.543           0.457
x4          0.844*                   0.279           0.721
x5          0.895*                   0.243           0.757
x6          0.810*                   0.305           0.695
x7      .*          0.775*           0.502           0.498
x8                  0.744*           0.469           0.531
x9  0.312*          0.481*           0.543           0.457

                              f2    f3    f1 total
Sum of sq (obliq) loadings 2.215 1.343 1.297 4.855
Proportion var             0.246 0.149 0.144 0.539
Cumulative var             0.246 0.395 0.539 0.539
Proportion of total        0.456 0.277 0.267 1.000

Factor correlations:

      f1    f2    f3
f1 1.000            
f2 0.373 1.000      
f3 0.432 0.306 1.000
\end{verbatim}
