\hypertarget{estimators}{%
\paragraph{Estimators}\label{estimators}}

If all data is continuous, the default estimator in the lavaan package
is maximum likelihood (\texttt{estimator\ =\ "ML"}). Alternative
estimators available in lavaan are:

\begin{itemize}
\tightlist
\item
  \texttt{"GLS"}: generalized least squares. For complete data only.
\item
  \texttt{"WLS"}: weighted least squares (sometimes called ADF
  estimation). For complete data only.
\item
  \texttt{"DWLS"}: diagonally weighted least squares
\item
  \texttt{"ULS"}: unweighted least squares
\end{itemize}

Many estimators have `robust' variants, meaning that they provide robust
standard errors and a scaled test statistic. For example, for the
maximum likelihood estimator, lavaan provides the following robust
variants:

\begin{itemize}
\tightlist
\item
  \texttt{"MLM"}: maximum likelihood estimation with robust standard
  errors and a Satorra-Bentler scaled test statistic. For complete data
  only.
\item
  \texttt{"MLMVS"}: maximum likelihood estimation with robust standard
  errors and a mean- and variance adjusted test statistic (aka the
  Satterthwaite approach). For complete data only.
\item
  \texttt{"MLMV"}: maximum likelihood estimation with robust standard
  errors and a mean- and variance adjusted test statistic (using a
  scale-shifted approach). For complete data only.
\item
  \texttt{"MLF"}: for maximum likelihood estimation with standard errors
  based on the first-order derivatives, and a conventional test
  statistic. For both complete and incomplete data.
\item
  \texttt{"MLR"}: maximum likelihood estimation with robust
  (Huber-White) standard errors and a scaled test statistic that is
  (asymptotically) equal to the Yuan-Bentler test statistic. For both
  complete and incomplete data.
\end{itemize}

For the \texttt{DWLS} and \texttt{ULS} estimators, lavaan also provides
`robust' variants: \texttt{WLSM}, \texttt{WLSMVS}, \texttt{WLSMV},
\texttt{ULSM}, \texttt{ULSMVS}, \texttt{ULSMV}. Note that for the robust
\texttt{WLS} variants, we use the diagonal of the weight matrix for
estimation, but we use the full weight matrix to correct the standard
errors and to compute the test statistic.

\hypertarget{ml-estimation-wishart-versus-normal}{%
\paragraph{ML estimation: Wishart versus
Normal}\label{ml-estimation-wishart-versus-normal}}

If maximum likelihood estimation is used (\texttt{"ML"} or any of its
robusts variants), the default behavior of lavaan is to base the
analysis on the so-called \emph{biased} sample covariance matrix, where
the elements are divided by \(n\) instead of \(n-1\). This is done
internally, and should not be done by the user. In addition, the
chi-square statistic is computed by multiplying the minimum function
value with a factor \(n\) (instead of \(n-1\)). This is similar to the
Mplus program. If you prefer to use an unbiased covariance, and \(n-1\)
as the multiplier to compute the chi-square statistic, you need to
specify the \texttt{likelihood\ =\ "wishart"} argument when calling the
fitting functions. For example:

\begin{Shaded}
\begin{Highlighting}[]
\NormalTok{fit <-}\StringTok{ }\KeywordTok{cfa}\NormalTok{(HS.model, }
           \DataTypeTok{data =}\NormalTok{ HolzingerSwineford1939, }
           \DataTypeTok{likelihood =} \StringTok{"wishart"}\NormalTok{)}
\NormalTok{fit}
\end{Highlighting}
\end{Shaded}

\begin{verbatim}
lavaan (0.6-1) converged normally after  35 iterations

  Number of observations                           301

  Estimator                                         ML
  Model Fit Test Statistic                      85.022
  Degrees of freedom                                24
  P-value (Chi-square)                           0.000
\end{verbatim}

The value of the test statistic will be closer to the value reported by
programs like EQS, LISREL or AMOS, since they all use the `Wishart'
approach when using the maximum likelihood estimator. The program Mplus,
on the other hand, uses the `normal' approach to maximum likelihood
estimation.

\hypertarget{missing-values}{%
\paragraph{Missing values}\label{missing-values}}

If the data contain missing values, the default behavior is listwise
deletion. If the missing mechanism is MCAR (missing completely at
random) or MAR (missing at random), the lavaan package provides
case-wise (or `full information') maximum likelihood estimation. You can
turn this feature on, by using the argument \texttt{missing\ =\ "ML"}
when calling the fitting function. An unrestricted (h1) model will
automatically be estimated, so that all common fit indices are
available.

\hypertarget{standard-errors}{%
\paragraph{Standard errors}\label{standard-errors}}

Standard errors are (by default) based on the expected information
matrix. The only exception is when data are missing and full information
ML is used (via \texttt{missing\ =\ "ML"}). In this case, the observed
information matrix is used to compute the standard errors. The user can
change this behavior by using the \texttt{information} argument, which
can be set to \texttt{"expected"} or \texttt{"observed"}.

If the estimator is simply \texttt{"ML"}, you can request robust
standard errors by using the \texttt{se} argument, which can be set to
\texttt{"robust.sem"}, \texttt{"robust.huber.white"} ,
\texttt{"first.order"} or \texttt{"bootstrap"}.\\
Or simply to \texttt{"none"} if you don't need them. This will not
affect the test statistic. In fact, you can choose the test statistic
independently by using the \texttt{test} argument, which can be set to
\texttt{"standard"}, \texttt{"Satorra-Bentler"}, \texttt{"Yuan-Bentler"}
or \texttt{"bootstrap"}.

\hypertarget{bootstrapping}{%
\paragraph{Bootstrapping}\label{bootstrapping}}

There are two ways for using the bootstrap in lavaan. Either you can set
\texttt{se\ =\ "bootstrap"} or \texttt{test\ =\ "bootstrap"} when
fitting the model (and you will get bootstrap standard errors, and/or a
bootstrap based p-value respectively), or you can you the
\texttt{bootstrapLavaan()} function, which needs an already fitted
lavaan object. The latter function can be used to `bootstrap' any
statistic (or vector of statistics) that you can extract from a fitted
lavaan object.
