The \texttt{summary()} function gives a nice overview of a fitted model,
but is for display only. If you need the actual numbers for further
processing, you may prefer to use one of several `extractor' functions.
We have already seen the \texttt{coef()} function which extracts the
estimated parameters of a fitted model. Other extractor functions are
discussed below.

\hypertarget{parameterestimates}{%
\paragraph{parameterEstimates}\label{parameterestimates}}

The \texttt{parameterEstimates()} function returns a data.frame
containing all the model parameters in the rows:

\begin{Shaded}
\begin{Highlighting}[]
\NormalTok{fit }\OtherTok{\textless{}{-}} \FunctionTok{cfa}\NormalTok{(HS.model, }\AttributeTok{data=}\NormalTok{HolzingerSwineford1939)}
\FunctionTok{parameterEstimates}\NormalTok{(fit)}
\end{Highlighting}
\end{Shaded}

\begin{verbatim}
       lhs op     rhs   est    se      z pvalue ci.lower ci.upper
1   visual =~      x1 1.000 0.000     NA     NA    1.000    1.000
2   visual =~      x2 0.554 0.100  5.554      0    0.358    0.749
3   visual =~      x3 0.729 0.109  6.685      0    0.516    0.943
4  textual =~      x4 1.000 0.000     NA     NA    1.000    1.000
5  textual =~      x5 1.113 0.065 17.014      0    0.985    1.241
6  textual =~      x6 0.926 0.055 16.703      0    0.817    1.035
7    speed =~      x7 1.000 0.000     NA     NA    1.000    1.000
8    speed =~      x8 1.180 0.165  7.152      0    0.857    1.503
9    speed =~      x9 1.082 0.151  7.155      0    0.785    1.378
10      x1 ~~      x1 0.549 0.114  4.833      0    0.326    0.772
11      x2 ~~      x2 1.134 0.102 11.146      0    0.934    1.333
12      x3 ~~      x3 0.844 0.091  9.317      0    0.667    1.022
13      x4 ~~      x4 0.371 0.048  7.779      0    0.278    0.465
14      x5 ~~      x5 0.446 0.058  7.642      0    0.332    0.561
15      x6 ~~      x6 0.356 0.043  8.277      0    0.272    0.441
16      x7 ~~      x7 0.799 0.081  9.823      0    0.640    0.959
17      x8 ~~      x8 0.488 0.074  6.573      0    0.342    0.633
18      x9 ~~      x9 0.566 0.071  8.003      0    0.427    0.705
19  visual ~~  visual 0.809 0.145  5.564      0    0.524    1.094
20 textual ~~ textual 0.979 0.112  8.737      0    0.760    1.199
21   speed ~~   speed 0.384 0.086  4.451      0    0.215    0.553
22  visual ~~ textual 0.408 0.074  5.552      0    0.264    0.552
23  visual ~~   speed 0.262 0.056  4.660      0    0.152    0.373
24 textual ~~   speed 0.173 0.049  3.518      0    0.077    0.270
\end{verbatim}

The \texttt{lhs} (left-hand side), \texttt{op} (operator) and
\texttt{rhs} (right-hand side) columns define the parameter. The
\texttt{est}, \texttt{se}, \texttt{z} and \texttt{pvalue} columns
provide the point estimate, the standard error, the z-value and the
p-value for this parameter. The last two columns are the lower and upper
bounds of a 95\% confidence interval around the point estimate.

\hypertarget{standardizedsolution}{%
\paragraph{standardizedSolution}\label{standardizedsolution}}

The \texttt{standardizedSolution()} function is similar to the
\texttt{parameterEstimates()} function, but only shows the standardized
parameter estimates and corresponding standard errors, z-values,
p-values and confidence intervals.

\hypertarget{fitted.values}{%
\paragraph{fitted.values}\label{fitted.values}}

The \texttt{fitted()} and \texttt{fitted.values()} functions return the
model-implied (fitted) covariance matrix (and mean vector) of a fitted
model:

\begin{Shaded}
\begin{Highlighting}[]
\NormalTok{fit }\OtherTok{\textless{}{-}} \FunctionTok{cfa}\NormalTok{(HS.model, }\AttributeTok{data =}\NormalTok{ HolzingerSwineford1939)}
\FunctionTok{fitted}\NormalTok{(fit)}
\end{Highlighting}
\end{Shaded}

\begin{verbatim}
$cov
   x1    x2    x3    x4    x5    x6    x7    x8    x9   
x1 1.358                                                
x2 0.448 1.382                                          
x3 0.590 0.327 1.275                                    
x4 0.408 0.226 0.298 1.351                              
x5 0.454 0.252 0.331 1.090 1.660                        
x6 0.378 0.209 0.276 0.907 1.010 1.196                  
x7 0.262 0.145 0.191 0.173 0.193 0.161 1.183            
x8 0.309 0.171 0.226 0.205 0.228 0.190 0.453 1.022      
x9 0.284 0.157 0.207 0.188 0.209 0.174 0.415 0.490 1.015
\end{verbatim}

\hypertarget{residuals}{%
\paragraph{residuals}\label{residuals}}

The \texttt{resid()} or \texttt{residuals()} functions return
(unstandardized) residuals of a fitted model. This is simply the
difference between the observed and implied covariance matrix and mean
vector.

\begin{Shaded}
\begin{Highlighting}[]
\NormalTok{fit }\OtherTok{\textless{}{-}} \FunctionTok{cfa}\NormalTok{(HS.model, }\AttributeTok{data =}\NormalTok{ HolzingerSwineford1939)}
\FunctionTok{resid}\NormalTok{(fit)}
\end{Highlighting}
\end{Shaded}

\begin{verbatim}
$type
[1] "raw"

$cov
   x1     x2     x3     x4     x5     x6     x7     x8     x9    
x1  0.000                                                        
x2 -0.041  0.000                                                 
x3 -0.010  0.124  0.000                                          
x4  0.097 -0.017 -0.090  0.000                                   
x5 -0.014 -0.040 -0.219  0.008  0.000                            
x6  0.077  0.038 -0.032 -0.012  0.005  0.000                     
x7 -0.177 -0.242 -0.103  0.046 -0.050 -0.017  0.000              
x8 -0.046 -0.062 -0.013 -0.079 -0.047 -0.024  0.082  0.000       
x9  0.175  0.087  0.167  0.056  0.086  0.062 -0.042 -0.032  0.000
\end{verbatim}

The \texttt{lavResiduals()} gives more extensive information about the
residuals. Per default, it will print both raw and standardized
residuals, as well as several summary statistics (including the SRMR and
the unbiased SRMR).

\hypertarget{vcov}{%
\paragraph{vcov}\label{vcov}}

The function \texttt{vcov()} returns the estimated covariance matrix of
the parameter estimates.

\hypertarget{aic-and-bic}{%
\paragraph{AIC and BIC}\label{aic-and-bic}}

The \texttt{AIC()} and \texttt{BIC()} functions return the AIC and BIC
values of a fitted model.

\hypertarget{fitmeasures}{%
\paragraph{fitMeasures}\label{fitmeasures}}

The \texttt{fitMeasures()} function returns all the fit measures
computed by lavaan as a named numeric vector.

\begin{Shaded}
\begin{Highlighting}[]
\NormalTok{fit }\OtherTok{\textless{}{-}} \FunctionTok{cfa}\NormalTok{(HS.model, }\AttributeTok{data=}\NormalTok{HolzingerSwineford1939)}
\FunctionTok{fitMeasures}\NormalTok{(fit)}
\end{Highlighting}
\end{Shaded}

\begin{verbatim}
               npar                fmin               chisq                  df 
             21.000               0.142              85.306              24.000 
             pvalue      baseline.chisq         baseline.df     baseline.pvalue 
              0.000             918.852              36.000               0.000 
                cfi                 tli                nnfi                 rfi 
              0.931               0.896               0.896               0.861 
                nfi                pnfi                 ifi                 rni 
              0.907               0.605               0.931               0.931 
               logl   unrestricted.logl                 aic                 bic 
          -3737.745           -3695.092            7517.490            7595.339 
             ntotal                bic2               rmsea      rmsea.ci.lower 
            301.000            7528.739               0.092               0.071 
     rmsea.ci.upper        rmsea.pvalue                 rmr          rmr_nomean 
              0.114               0.001               0.082               0.082 
               srmr        srmr_bentler srmr_bentler_nomean                crmr 
              0.065               0.065               0.065               0.073 
        crmr_nomean          srmr_mplus   srmr_mplus_nomean               cn_05 
              0.073               0.065               0.065             129.490 
              cn_01                 gfi                agfi                pgfi 
            152.654               0.943               0.894               0.503 
                mfi                ecvi 
              0.903               0.423 
\end{verbatim}

If you only want the value of a single fit measure, say, the CFI, you
give the name (in lower case) as the second argument:

\begin{Shaded}
\begin{Highlighting}[]
\NormalTok{fit }\OtherTok{\textless{}{-}} \FunctionTok{cfa}\NormalTok{(HS.model, }\AttributeTok{data=}\NormalTok{HolzingerSwineford1939)}
\FunctionTok{fitMeasures}\NormalTok{(fit, }\StringTok{"cfi"}\NormalTok{)}
\end{Highlighting}
\end{Shaded}

\begin{verbatim}
  cfi 
0.931 
\end{verbatim}

Or you can provide a vector of fit measures, as in

\begin{Shaded}
\begin{Highlighting}[]
\FunctionTok{fitMeasures}\NormalTok{(fit, }\FunctionTok{c}\NormalTok{(}\StringTok{"cfi"}\NormalTok{,}\StringTok{"rmsea"}\NormalTok{,}\StringTok{"srmr"}\NormalTok{))}
\end{Highlighting}
\end{Shaded}

\begin{verbatim}
  cfi rmsea  srmr 
0.931 0.092 0.065 
\end{verbatim}

\hypertarget{lavinspect}{%
\paragraph{lavInspect}\label{lavinspect}}

If you want to peek inside a fitted lavaan object (the object that is
returned by a call to \texttt{cfa()}, \texttt{sem()}or
\texttt{growth()}), you can use the \texttt{lavInspect()} function, with
a variety of options. By default, calling \texttt{lavInspect()} on a
fitted lavaan object returns a list of the model matrices that are used
internally to represent the model. The free parameters are nonzero
integers.

\begin{Shaded}
\begin{Highlighting}[]
\NormalTok{fit }\OtherTok{\textless{}{-}} \FunctionTok{cfa}\NormalTok{(HS.model, }\AttributeTok{data=}\NormalTok{HolzingerSwineford1939)}
\FunctionTok{lavInspect}\NormalTok{(fit)}
\end{Highlighting}
\end{Shaded}

\begin{verbatim}
$lambda
   visual textul speed
x1      0      0     0
x2      1      0     0
x3      2      0     0
x4      0      0     0
x5      0      3     0
x6      0      4     0
x7      0      0     0
x8      0      0     5
x9      0      0     6

$theta
   x1 x2 x3 x4 x5 x6 x7 x8 x9
x1  7                        
x2  0  8                     
x3  0  0  9                  
x4  0  0  0 10               
x5  0  0  0  0 11            
x6  0  0  0  0  0 12         
x7  0  0  0  0  0  0 13      
x8  0  0  0  0  0  0  0 14   
x9  0  0  0  0  0  0  0  0 15

$psi
        visual textul speed
visual  16                 
textual 19     17          
speed   20     21     18   
\end{verbatim}

To see the starting values of parameters in each model matrix, type

\begin{Shaded}
\begin{Highlighting}[]
\FunctionTok{lavInspect}\NormalTok{(fit, }\AttributeTok{what =} \StringTok{"start"}\NormalTok{)}
\end{Highlighting}
\end{Shaded}

\begin{verbatim}
$lambda
   visual textul speed
x1  1.000  0.000 0.000
x2  0.778  0.000 0.000
x3  1.107  0.000 0.000
x4  0.000  1.000 0.000
x5  0.000  1.133 0.000
x6  0.000  0.924 0.000
x7  0.000  0.000 1.000
x8  0.000  0.000 1.225
x9  0.000  0.000 0.854

$theta
   x1    x2    x3    x4    x5    x6    x7    x8    x9   
x1 0.679                                                
x2 0.000 0.691                                          
x3 0.000 0.000 0.637                                    
x4 0.000 0.000 0.000 0.675                              
x5 0.000 0.000 0.000 0.000 0.830                        
x6 0.000 0.000 0.000 0.000 0.000 0.598                  
x7 0.000 0.000 0.000 0.000 0.000 0.000 0.592            
x8 0.000 0.000 0.000 0.000 0.000 0.000 0.000 0.511      
x9 0.000 0.000 0.000 0.000 0.000 0.000 0.000 0.000 0.508

$psi
        visual textul speed
visual  0.05               
textual 0.00   0.05        
speed   0.00   0.00   0.05 
\end{verbatim}

To see how lavaan internally represents a model, you can type

\begin{Shaded}
\begin{Highlighting}[]
\FunctionTok{lavInspect}\NormalTok{(fit, }\AttributeTok{what =} \StringTok{"list"}\NormalTok{)}
\end{Highlighting}
\end{Shaded}

\begin{verbatim}
   id     lhs op     rhs user block group free ustart exo label plabel start
1   1  visual =~      x1    1     1     1    0      1   0         .p1. 1.000
2   2  visual =~      x2    1     1     1    1     NA   0         .p2. 0.778
3   3  visual =~      x3    1     1     1    2     NA   0         .p3. 1.107
4   4 textual =~      x4    1     1     1    0      1   0         .p4. 1.000
5   5 textual =~      x5    1     1     1    3     NA   0         .p5. 1.133
6   6 textual =~      x6    1     1     1    4     NA   0         .p6. 0.924
7   7   speed =~      x7    1     1     1    0      1   0         .p7. 1.000
8   8   speed =~      x8    1     1     1    5     NA   0         .p8. 1.225
9   9   speed =~      x9    1     1     1    6     NA   0         .p9. 0.854
10 10      x1 ~~      x1    0     1     1    7     NA   0        .p10. 0.679
11 11      x2 ~~      x2    0     1     1    8     NA   0        .p11. 0.691
12 12      x3 ~~      x3    0     1     1    9     NA   0        .p12. 0.637
13 13      x4 ~~      x4    0     1     1   10     NA   0        .p13. 0.675
14 14      x5 ~~      x5    0     1     1   11     NA   0        .p14. 0.830
15 15      x6 ~~      x6    0     1     1   12     NA   0        .p15. 0.598
16 16      x7 ~~      x7    0     1     1   13     NA   0        .p16. 0.592
17 17      x8 ~~      x8    0     1     1   14     NA   0        .p17. 0.511
18 18      x9 ~~      x9    0     1     1   15     NA   0        .p18. 0.508
19 19  visual ~~  visual    0     1     1   16     NA   0        .p19. 0.050
20 20 textual ~~ textual    0     1     1   17     NA   0        .p20. 0.050
21 21   speed ~~   speed    0     1     1   18     NA   0        .p21. 0.050
22 22  visual ~~ textual    0     1     1   19     NA   0        .p22. 0.000
23 23  visual ~~   speed    0     1     1   20     NA   0        .p23. 0.000
24 24 textual ~~   speed    0     1     1   21     NA   0        .p24. 0.000
     est    se
1  1.000 0.000
2  0.554 0.100
3  0.729 0.109
4  1.000 0.000
5  1.113 0.065
6  0.926 0.055
7  1.000 0.000
8  1.180 0.165
9  1.082 0.151
10 0.549 0.114
11 1.134 0.102
12 0.844 0.091
13 0.371 0.048
14 0.446 0.058
15 0.356 0.043
16 0.799 0.081
17 0.488 0.074
18 0.566 0.071
19 0.809 0.145
20 0.979 0.112
21 0.384 0.086
22 0.408 0.074
23 0.262 0.056
24 0.173 0.049
\end{verbatim}

This is equivalent to the \texttt{parTable(fit)} function. The table
that is returned here is called the `parameter table'.

For more lavInspect options, see the help page:

\begin{Shaded}
\begin{Highlighting}[]
\NormalTok{?lavInspect}
\end{Highlighting}
\end{Shaded}
