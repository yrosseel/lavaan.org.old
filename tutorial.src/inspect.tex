The \texttt{summary()} function gives a nice overview of a fitted model,
but is for display only. If you need the actual numbers for further
processing, you may prefer to use one of several `extractor' functions.
We have already seen the \texttt{coef()} function which extracts the
estimated parameters of a fitted model. Other extractor functions are
discussed below.

\paragraph{parameterEstimates}

The \texttt{parameterEstimates} function extracts not only the values of
the estimated parameters, but also the standard errors, the z-values,
the standardized parameter values, and returns everything conveniently
as a data frame. For example:

\begin{Shaded}
\begin{Highlighting}[]
\NormalTok{fit <-}\StringTok{ }\KeywordTok{cfa}\NormalTok{(HS.model, }\DataTypeTok{data =} \NormalTok{HolzingerSwineford1939)}
\KeywordTok{parameterEstimates}\NormalTok{(fit)}
\end{Highlighting}
\end{Shaded}

\begin{verbatim}
       lhs op     rhs   est    se      z pvalue ci.lower ci.upper
1   visual =~      x1 1.000 0.000     NA     NA    1.000    1.000
2   visual =~      x2 0.553 0.100  5.554      0    0.358    0.749
3   visual =~      x3 0.729 0.109  6.685      0    0.516    0.943
4  textual =~      x4 1.000 0.000     NA     NA    1.000    1.000
5  textual =~      x5 1.113 0.065 17.014      0    0.985    1.241
6  textual =~      x6 0.926 0.055 16.703      0    0.817    1.035
7    speed =~      x7 1.000 0.000     NA     NA    1.000    1.000
8    speed =~      x8 1.180 0.165  7.152      0    0.857    1.503
9    speed =~      x9 1.082 0.151  7.155      0    0.785    1.378
10      x1 ~~      x1 0.549 0.114  4.833      0    0.326    0.772
11      x2 ~~      x2 1.134 0.102 11.146      0    0.934    1.333
12      x3 ~~      x3 0.844 0.091  9.317      0    0.667    1.022
13      x4 ~~      x4 0.371 0.048  7.779      0    0.278    0.465
14      x5 ~~      x5 0.446 0.058  7.642      0    0.332    0.561
15      x6 ~~      x6 0.356 0.043  8.277      0    0.272    0.441
16      x7 ~~      x7 0.799 0.081  9.823      0    0.640    0.959
17      x8 ~~      x8 0.488 0.074  6.573      0    0.342    0.633
18      x9 ~~      x9 0.566 0.071  8.003      0    0.427    0.705
19  visual ~~  visual 0.809 0.145  5.564      0    0.524    1.094
20 textual ~~ textual 0.979 0.112  8.737      0    0.760    1.199
21   speed ~~   speed 0.384 0.086  4.451      0    0.215    0.553
22  visual ~~ textual 0.408 0.074  5.552      0    0.264    0.552
23  visual ~~   speed 0.262 0.056  4.660      0    0.152    0.373
24 textual ~~   speed 0.173 0.049  3.518      0    0.077    0.270
\end{verbatim}

\paragraph{standardizedSolution}

The \texttt{standardizedSolution()} function is similar to the
\texttt{parameterEstimates()} function, but only shows the
unstandardized and standardized parameter estimates.

\paragraph{fitted.values}

The \texttt{fitted()} and \texttt{fitted.values()} functions return the
model-implied (fitted) covariance matrix (and mean vector) of a fitted
model:

\begin{Shaded}
\begin{Highlighting}[]
\NormalTok{fit <-}\StringTok{ }\KeywordTok{cfa}\NormalTok{(HS.model, }\DataTypeTok{data =} \NormalTok{HolzingerSwineford1939)}
\KeywordTok{fitted}\NormalTok{(fit)}
\end{Highlighting}
\end{Shaded}

\begin{verbatim}
$cov
   x1    x2    x3    x4    x5    x6    x7    x8    x9   
x1 1.358                                                
x2 0.448 1.382                                          
x3 0.590 0.327 1.275                                    
x4 0.408 0.226 0.298 1.351                              
x5 0.454 0.252 0.331 1.090 1.660                        
x6 0.378 0.209 0.276 0.907 1.010 1.196                  
x7 0.262 0.145 0.191 0.173 0.193 0.161 1.183            
x8 0.309 0.171 0.226 0.205 0.228 0.190 0.453 1.022      
x9 0.284 0.157 0.207 0.188 0.209 0.174 0.415 0.490 1.015

$mean
x1 x2 x3 x4 x5 x6 x7 x8 x9 
 0  0  0  0  0  0  0  0  0 
\end{verbatim}

\paragraph{residuals}

The \texttt{resid()} or \texttt{residuals()} functions return
(unstandardized) residuals of a fitted model. This is simply the
difference between the observed and implied covariance matrix and mean
vector. If the estimator is maximum likelihood, it is also possible to
obtain the normalized and the standardized residuals (note: you may
observe several \texttt{NA} values, but they can be safely ignored)

\begin{Shaded}
\begin{Highlighting}[]
\NormalTok{fit <-}\StringTok{ }\KeywordTok{cfa}\NormalTok{(HS.model, }\DataTypeTok{data =} \NormalTok{HolzingerSwineford1939)}
\KeywordTok{resid}\NormalTok{(fit, }\DataTypeTok{type =} \StringTok{"standardized"}\NormalTok{)}
\end{Highlighting}
\end{Shaded}

\begin{verbatim}
$cov
   x1     x2     x3     x4     x5     x6     x7     x8     x9    
x1     NA                                                        
x2 -2.196     NA                                                 
x3 -1.199  2.692  0.000                                          
x4  2.465 -0.283 -1.948     NA                                   
x5 -0.362 -0.610 -4.443  0.856     NA                            
x6  2.032  0.661 -0.701     NA  0.633     NA                     
x7 -3.787 -3.800 -1.882  0.839 -0.837 -0.321  0.000              
x8 -1.456 -1.137 -0.305 -2.049 -1.100 -0.635  3.804     NA       
x9  4.062  1.517  3.328  1.237  1.723  1.436 -2.772     NA     NA

$mean
x1 x2 x3 x4 x5 x6 x7 x8 x9 
 0  0  0  0  0  0  0  0  0 
\end{verbatim}

\paragraph{vcov}

The function \texttt{vcov()} returns the estimated covariance matrix of
the parameter estimates.

\paragraph{AIC and BIC}

The \texttt{AIC()} and \texttt{BIC()} functions return the AIC and BIC
values of a fitted model.

\paragraph{fitMeasures}

The \texttt{fitMeasures()} function returns all the fit measures
computed by lavaan as a named numeric vector.

\begin{Shaded}
\begin{Highlighting}[]
\NormalTok{fit <-}\StringTok{ }\KeywordTok{cfa}\NormalTok{(HS.model, }\DataTypeTok{data =} \NormalTok{HolzingerSwineford1939)}
\KeywordTok{fitMeasures}\NormalTok{(fit)}
\end{Highlighting}
\end{Shaded}

\begin{verbatim}
             fmin             chisq                df            pvalue 
            0.142            85.306            24.000             0.000 
   baseline.chisq       baseline.df   baseline.pvalue               cfi 
          918.852            36.000             0.000             0.931 
              tli              nnfi               rfi               nfi 
            0.896             0.896             0.861             0.907 
             pnfi               ifi               rni              logl 
            0.605             0.931             0.931         -3737.745 
unrestricted.logl              npar               aic               bic 
        -3695.092            21.000          7517.490          7595.339 
           ntotal              bic2             rmsea    rmsea.ci.lower 
          301.000          7528.739             0.092             0.071 
   rmsea.ci.upper      rmsea.pvalue               rmr        rmr_nomean 
            0.114             0.001             0.082             0.082 
             srmr       srmr_nomean             cn_05             cn_01 
            0.065             0.065           129.490           152.654 
              gfi              agfi              pgfi               mfi 
            0.943             0.894             0.503             0.903 
             ecvi 
            0.423 
\end{verbatim}

If you only want the value of a single fit measure, say, the CFI, you
give the name (in lower case) as the second argument:

\begin{Shaded}
\begin{Highlighting}[]
\NormalTok{fit <-}\StringTok{ }\KeywordTok{cfa}\NormalTok{(HS.model, }\DataTypeTok{data =} \NormalTok{HolzingerSwineford1939)}
\KeywordTok{fitMeasures}\NormalTok{(fit, }\StringTok{"cfi"}\NormalTok{)}
\end{Highlighting}
\end{Shaded}

\begin{verbatim}
  cfi 
0.931 
\end{verbatim}

Or you can provide a vector of fit measures, as in

\begin{Shaded}
\begin{Highlighting}[]
\KeywordTok{fitMeasures}\NormalTok{(fit, }\KeywordTok{c}\NormalTok{(}\StringTok{"cfi"}\NormalTok{, }\StringTok{"rmsea"}\NormalTok{, }\StringTok{"srmr"}\NormalTok{))}
\end{Highlighting}
\end{Shaded}

\begin{verbatim}
  cfi rmsea  srmr 
0.931 0.092 0.065 
\end{verbatim}

\paragraph{inspect}

If you want to peek inside a fitted lavaan object (the object that is
returned by a call to \texttt{cfa()}, \texttt{sem()}or
\texttt{growth()}), you can use the \texttt{inspect()} function, with a
variety of options. By default, calling \texttt{inspect()} on a fitted
lavaan object returns a list of the model matrices that are used
internally to represent the model. The free parameters are nonzero
integers.

\begin{Shaded}
\begin{Highlighting}[]
\NormalTok{fit <-}\StringTok{ }\KeywordTok{cfa}\NormalTok{(HS.model, }\DataTypeTok{data =} \NormalTok{HolzingerSwineford1939)}
\KeywordTok{inspect}\NormalTok{(fit)}
\end{Highlighting}
\end{Shaded}

\begin{verbatim}
$lambda
   visual textul speed
x1      0      0     0
x2      1      0     0
x3      2      0     0
x4      0      0     0
x5      0      3     0
x6      0      4     0
x7      0      0     0
x8      0      0     5
x9      0      0     6

$theta
   x1 x2 x3 x4 x5 x6 x7 x8 x9
x1  7                        
x2  0  8                     
x3  0  0  9                  
x4  0  0  0 10               
x5  0  0  0  0 11            
x6  0  0  0  0  0 12         
x7  0  0  0  0  0  0 13      
x8  0  0  0  0  0  0  0 14   
x9  0  0  0  0  0  0  0  0 15

$psi
        visual textul speed
visual  16                 
textual 19     17          
speed   20     21     18   
\end{verbatim}

To see the starting values of parameters in each model matrix, type

\begin{Shaded}
\begin{Highlighting}[]
\KeywordTok{inspect}\NormalTok{(fit, }\DataTypeTok{what =} \StringTok{"start"}\NormalTok{)}
\end{Highlighting}
\end{Shaded}

\begin{verbatim}
$lambda
   visual textul speed
x1  1.000  0.000 0.000
x2  0.778  0.000 0.000
x3  1.107  0.000 0.000
x4  0.000  1.000 0.000
x5  0.000  1.133 0.000
x6  0.000  0.924 0.000
x7  0.000  0.000 1.000
x8  0.000  0.000 1.225
x9  0.000  0.000 0.854

$theta
   x1    x2    x3    x4    x5    x6    x7    x8    x9   
x1 0.679                                                
x2 0.000 0.691                                          
x3 0.000 0.000 0.637                                    
x4 0.000 0.000 0.000 0.675                              
x5 0.000 0.000 0.000 0.000 0.830                        
x6 0.000 0.000 0.000 0.000 0.000 0.598                  
x7 0.000 0.000 0.000 0.000 0.000 0.000 0.592            
x8 0.000 0.000 0.000 0.000 0.000 0.000 0.000 0.511      
x9 0.000 0.000 0.000 0.000 0.000 0.000 0.000 0.000 0.508

$psi
        visual textul speed
visual  0.05               
textual 0.00   0.05        
speed   0.00   0.00   0.05 
\end{verbatim}

To see how lavaan internally represents a model, you can type

\begin{Shaded}
\begin{Highlighting}[]
\KeywordTok{inspect}\NormalTok{(fit, }\DataTypeTok{what =} \StringTok{"list"}\NormalTok{)}
\end{Highlighting}
\end{Shaded}

\begin{verbatim}
   id     lhs op     rhs user group free ustart exo label eq.id unco
1   1  visual =~      x1    1     1    0      1   0           0    0
2   2  visual =~      x2    1     1    1     NA   0           0    1
3   3  visual =~      x3    1     1    2     NA   0           0    2
4   4 textual =~      x4    1     1    0      1   0           0    0
5   5 textual =~      x5    1     1    3     NA   0           0    3
6   6 textual =~      x6    1     1    4     NA   0           0    4
7   7   speed =~      x7    1     1    0      1   0           0    0
8   8   speed =~      x8    1     1    5     NA   0           0    5
9   9   speed =~      x9    1     1    6     NA   0           0    6
10 10      x1 ~~      x1    0     1    7     NA   0           0    7
11 11      x2 ~~      x2    0     1    8     NA   0           0    8
12 12      x3 ~~      x3    0     1    9     NA   0           0    9
13 13      x4 ~~      x4    0     1   10     NA   0           0   10
14 14      x5 ~~      x5    0     1   11     NA   0           0   11
15 15      x6 ~~      x6    0     1   12     NA   0           0   12
16 16      x7 ~~      x7    0     1   13     NA   0           0   13
17 17      x8 ~~      x8    0     1   14     NA   0           0   14
18 18      x9 ~~      x9    0     1   15     NA   0           0   15
19 19  visual ~~  visual    0     1   16     NA   0           0   16
20 20 textual ~~ textual    0     1   17     NA   0           0   17
21 21   speed ~~   speed    0     1   18     NA   0           0   18
22 22  visual ~~ textual    0     1   19     NA   0           0   19
23 23  visual ~~   speed    0     1   20     NA   0           0   20
24 24 textual ~~   speed    0     1   21     NA   0           0   21
\end{verbatim}

This is equivalent to the \texttt{parTable(fit)} function. The table
that is returned here is called the `parameter table'.

For more inspect options, see the help page for the lavaan class which
you can find by typing the following:

\begin{Shaded}
\begin{Highlighting}[]
\NormalTok{class?lavaan}
\end{Highlighting}
\end{Shaded}

