Before you start, please read these points carefully:

\begin{itemize}
\item
  First of all, you must have a recent version (4.0.0 or higher) of R
  installed. You can download the latest version of R from this page:
  \url{http://cran.r-project.org/}.
\item
  Some important features are NOT available (yet) in lavaan:

  \begin{itemize}
  \item
    multilevel sem with random slopes (this is under developement)
  \item
    support for variable types other than continuous, binary and ordinal
    (for example: zero-inflated count data, nominal data, non-Gaussian
    continuous data); it is unlikely that this will be part of lavaan
    any time soon, for the simple reason that these variable types need
    numerical quadrature, and this is too slow to be practical in (pure)
    R.
  \item
    support for discrete latent variables (mixture models, latent
    classes) (although you can use the sampling weights and multiple
    group features to mimic some mixture models)
  \end{itemize}

  We hope to add these features to lavaan in the near future (but please
  do not ask when).
\item
  The lavaan package is free open-source software. This means (among
  other things) that there is no warranty whatsoever. On the other hand,
  you can verify the source code yourself:
  \url{https://github.com/yrosseel/lavaan/}
\item
  If you need help, you can (only) ask questions in the lavaan
  discussion group. Go to
  \url{https://groups.google.com/d/forum/lavaan/} and join the group.
  Once you have joined the group, you can email your questions to
  \href{mailto:lavaan@googlegroups.com}{\nolinkurl{lavaan@googlegroups.com}}.
  Please do not email me directly.
\item
  I do not offer statistical advice. For general (non lavaan-specific)
  questions about SEM, consider posting to the SEMNET discussion group.
\item
  If you think you have found a bug, or if you have a suggestion for
  improvement, you can either email me directly, or open an issue on
  github (see \url{https://github.com/yrosseel/lavaan/issues}). If you
  report a bug, always provide a minimal reproducible example (a short R
  script and some data).
\end{itemize}
