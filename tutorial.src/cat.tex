Binary, ordinal and nominal variables are considered categorical (not
continuous). It makes a big difference if these categorical variables
are exogenous (independent) or endogenous (dependent) in the model.

\hypertarget{exogenous-categorical-variables}{%
\paragraph{Exogenous categorical
variables}\label{exogenous-categorical-variables}}

If you have a binary exogenous covariate (say, gender), all you need to
do is to recode it as a dummy (0/1) variable. Just like you would do in
a classic regression model. If you have an exogenous ordinal variable,
you can use a coding scheme reflecting the order (say, 1,2,3,\ldots) and
treat it as any other (numeric) covariate. If you have a nominal
categorical variable with \(K > 2\) levels, you need to replace it by a
set of \(K-1\) dummy variables, again, just like you would do in
classical regression.

\hypertarget{endogenous-categorical-variables}{%
\paragraph{Endogenous categorical
variables}\label{endogenous-categorical-variables}}

The lavaan 0.5 series can deal with binary and ordinal (but not nominal)
endogenous variables. There are two ways to communicate to lavaan that
some of the endogenous variables are to be treated as categorical:

\begin{enumerate}
\def\labelenumi{\arabic{enumi}.}
\item
  declare them as `ordered' (using the \texttt{ordered} function, which
  is part of base R) in your data.frame before you run the analysis; for
  example, if you need to declare four variables (say, \texttt{item1},
  \texttt{item2}, \texttt{item3}, \texttt{item4}) as ordinal in your
  data.frame (called \texttt{Data}), you can use something like:

\begin{Shaded}
\begin{Highlighting}[]
\NormalTok{Data[,}\FunctionTok{c}\NormalTok{(}\StringTok{"item1"}\NormalTok{,}
        \StringTok{"item2"}\NormalTok{,}
        \StringTok{"item3"}\NormalTok{,}
        \StringTok{"item4"}\NormalTok{)] }\OtherTok{\textless{}{-}}
    \FunctionTok{lapply}\NormalTok{(Data[,}\FunctionTok{c}\NormalTok{(}\StringTok{"item1"}\NormalTok{,}
                   \StringTok{"item2"}\NormalTok{,}
                   \StringTok{"item3"}\NormalTok{,}
                   \StringTok{"item4"}\NormalTok{)], ordered)}
\end{Highlighting}
\end{Shaded}
\item
  use the \texttt{ordered} argument when using one of the fitting
  functions (cfa/sem/growth/lavaan), for example, if you have four
  binary or ordinal variables (say, \texttt{item1}, \texttt{item2},
  \texttt{item3}, \texttt{item4}), you can use:

\begin{Shaded}
\begin{Highlighting}[]
\NormalTok{fit }\OtherTok{\textless{}{-}} \FunctionTok{cfa}\NormalTok{(myModel, }\AttributeTok{data =}\NormalTok{ myData,}
           \AttributeTok{ordered =} \FunctionTok{c}\NormalTok{(}\StringTok{"item1"}\NormalTok{,}\StringTok{"item2"}\NormalTok{,}
                       \StringTok{"item3"}\NormalTok{,}\StringTok{"item4"}\NormalTok{))}
\end{Highlighting}
\end{Shaded}
\end{enumerate}

If all the (endogenous) variables are to be treated as categorical, you
can use \texttt{ordered\ =\ TRUE} as a shortcut.

When the \texttt{ordered=} argument is used, lavaan will automatically
switch to the \texttt{WLSMV} estimator: it will use diagonally weighted
least squares (\texttt{DWLS}) to estimate the model parameters, but it
will use the full weight matrix to compute robust standard errors, and a
mean- and variance-adjusted test stastistic. Other options are
unweighted least squares (ULSMV), or pairwise maximum likelihood (PML).
Full information maximum likelihood is currently not supported.
