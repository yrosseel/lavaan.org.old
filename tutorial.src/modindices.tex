Modification indices can be requested by adding the argument
\texttt{modindices\ =\ TRUE} in the \texttt{summary()} call, or by
calling the function \texttt{modindices()} directly. By default,
modification indices are printed out for each nonfree (or fixed-to-zero)
parameter. The modification indices are supplemented by the expected
parameter change (EPC) values (column \texttt{epc}). The last three
columns contain the standardized EPC values (\texttt{sepc.lv}: only
standardizing the latent variables; \texttt{sepc.all}: standardizing all
variables; \texttt{sepc.nox}: standardizing all but exogenous observed
variables).

A typical use of the \texttt{modindices()} function is as follows:

\begin{Shaded}
\begin{Highlighting}[]
\NormalTok{fit }\OtherTok{\textless{}{-}} \FunctionTok{cfa}\NormalTok{(HS.model,}
           \AttributeTok{data =}\NormalTok{ HolzingerSwineford1939)}
\FunctionTok{modindices}\NormalTok{(fit, }\AttributeTok{sort =} \ConstantTok{TRUE}\NormalTok{, }\AttributeTok{maximum.number =} \DecValTok{5}\NormalTok{)}
\end{Highlighting}
\end{Shaded}

\begin{verbatim}
       lhs op rhs     mi    epc sepc.lv sepc.all sepc.nox
30  visual =~  x9 36.411  0.577   0.519    0.515    0.515
76      x7 ~~  x8 34.145  0.536   0.536    0.859    0.859
28  visual =~  x7 18.631 -0.422  -0.380   -0.349   -0.349
78      x8 ~~  x9 14.946 -0.423  -0.423   -0.805   -0.805
33 textual =~  x3  9.151 -0.272  -0.269   -0.238   -0.238
\end{verbatim}

This will print out the top 5 parameters (that can be added to the
model) that result in the largest modification index, sorted from high
to low.

The \texttt{modindices()} function returns a data frame, which you can
sort or filter to extract what you want. For example, to see only the
modification indices for the factor loadings, you can use something like
this:

\begin{Shaded}
\begin{Highlighting}[]
\NormalTok{fit }\OtherTok{\textless{}{-}} \FunctionTok{cfa}\NormalTok{(HS.model, }
           \AttributeTok{data =}\NormalTok{ HolzingerSwineford1939)}
\NormalTok{mi }\OtherTok{\textless{}{-}} \FunctionTok{modindices}\NormalTok{(fit)}
\NormalTok{mi[mi}\SpecialCharTok{$}\NormalTok{op }\SpecialCharTok{==} \StringTok{"=\textasciitilde{}"}\NormalTok{,]}
\end{Highlighting}
\end{Shaded}

\begin{verbatim}
       lhs op rhs     mi    epc sepc.lv sepc.all sepc.nox
25  visual =~  x4  1.211  0.077   0.069    0.059    0.059
26  visual =~  x5  7.441 -0.210  -0.189   -0.147   -0.147
27  visual =~  x6  2.843  0.111   0.100    0.092    0.092
28  visual =~  x7 18.631 -0.422  -0.380   -0.349   -0.349
29  visual =~  x8  4.295 -0.210  -0.189   -0.187   -0.187
30  visual =~  x9 36.411  0.577   0.519    0.515    0.515
31 textual =~  x1  8.903  0.350   0.347    0.297    0.297
32 textual =~  x2  0.017 -0.011  -0.011   -0.010   -0.010
33 textual =~  x3  9.151 -0.272  -0.269   -0.238   -0.238
34 textual =~  x7  0.098 -0.021  -0.021   -0.019   -0.019
35 textual =~  x8  3.359 -0.121  -0.120   -0.118   -0.118
36 textual =~  x9  4.796  0.138   0.137    0.136    0.136
37   speed =~  x1  0.014  0.024   0.015    0.013    0.013
38   speed =~  x2  1.580 -0.198  -0.123   -0.105   -0.105
39   speed =~  x3  0.716  0.136   0.084    0.075    0.075
40   speed =~  x4  0.003 -0.005  -0.003   -0.003   -0.003
41   speed =~  x5  0.201 -0.044  -0.027   -0.021   -0.021
42   speed =~  x6  0.273  0.044   0.027    0.025    0.025
\end{verbatim}

It is important to realize that the \texttt{modindices()} function will
only consider fixed-to-zero parameters. If you have equality constraints
in the model, and you wish to examine what happens if you release all
(or some) of these equality constraints, use the \texttt{lavTestScore()}
function.
