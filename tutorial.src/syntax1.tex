At the heart of the lavaan package is the `model syntax'. The model
syntax is a description of the model to be estimated. In this section,
we briefly explain the elements of the lavaan model syntax. More details
are given in the examples that follow.

In the R environment, a regression formula has the following form:

\begin{verbatim}
y ~ x1 + x2 + x3 + x4
\end{verbatim}

In this formula, the tilde (``\texttt{\textasciitilde{}}'') is the
regression operator. On the left-hand side of the operator, we have the
dependent variable (\texttt{y}), and on the right-hand side, we have the
independent variables, separated by the ``\texttt{+}'' operator. In
lavaan, a typical model is simply a set (or system) of regression
formulas, where some variables (starting with an `\texttt{f}' below) may
be latent. For example:

\begin{verbatim}
 y ~ f1 + f2 + x1 + x2 
f1 ~ f2 + f3 
f2 ~ f3 + x1 + x2
\end{verbatim}

If we have latent variables in any of the regression formulas, we must
`define' them by listing their (manifest or latent) indicators. We do
this by using the special operator ``\texttt{=\textasciitilde{}}'',
which can be read as \emph{is measured by}. For example, to define the
three latent variabels \texttt{f1}, \texttt{f2} and \texttt{f3}, we can
use something like:

\begin{verbatim}
f1 =~ y1 + y2 + y3 
f2 =~ y4 + y5 + y6 
f3 =~ y7 + y8 + y9 + y10
\end{verbatim}

Furthermore, variances and covariances are specified using a `double
tilde' operator, for example:

\begin{verbatim}
y1 ~~ y1  # variance
y1 ~~ y2  # covariance
f1 ~~ f2  # covariance
\end{verbatim}

And finally, intercepts for observed and latent variables are simple
regression formulas with only an intercept (explicitly denoted by the
number `\texttt{1}') as the only predictor:

\begin{verbatim}
y1 ~ 1
f1 ~ 1
\end{verbatim}

Using these four \emph{formula types}, a large variety of latent
variable models can be described. The current set of formula types is
summarized in the table below.

\begin{longtable}[]{@{}lll@{}}
\toprule
formula type & operator & mnemonic \\
\midrule
\endhead
latent variable definition & \texttt{=\textasciitilde{}} & is measured
by \\
regression & \texttt{\textasciitilde{}} & is regressed on \\
(residual) (co)variance & \texttt{\textasciitilde{}\textasciitilde{}} &
is correlated with \\
intercept & \texttt{\textasciitilde{}\ 1} & intercept \\
\bottomrule
\end{longtable}

A complete lavaan model syntax is simply a combination of these formula
types, enclosed between \emph{single} quotes. For example:

\begin{Shaded}
\begin{Highlighting}[]
\NormalTok{myModel }\OtherTok{\textless{}{-}} \StringTok{\textquotesingle{} \# regressions}
\StringTok{             y1 + y2 \textasciitilde{} f1 + f2 + x1 + x2}
\StringTok{                  f1 \textasciitilde{} f2 + f3}
\StringTok{                  f2 \textasciitilde{} f3 + x1 + x2}

\StringTok{             \# latent variable definitions }
\StringTok{               f1 =\textasciitilde{} y1 + y2 + y3 }
\StringTok{               f2 =\textasciitilde{} y4 + y5 + y6 }
\StringTok{               f3 =\textasciitilde{} y7 + y8 + y9 + y10}

\StringTok{             \# variances and covariances }
\StringTok{               y1 \textasciitilde{}\textasciitilde{} y1 }
\StringTok{               y1 \textasciitilde{}\textasciitilde{} y2 }
\StringTok{               f1 \textasciitilde{}\textasciitilde{} f2}

\StringTok{             \# intercepts }
\StringTok{               y1 \textasciitilde{} 1 }
\StringTok{               f1 \textasciitilde{} 1}
\StringTok{           \textquotesingle{}}
\end{Highlighting}
\end{Shaded}

There reason why you should use single quotes is that this is the only
way (in R) to allow for double quotes inside a string. See
\texttt{?Quotes} in R for more information.

You can type this syntax interactively at the R prompt, but it is much
more convenient to type the whole model syntax first in an external text
editor. And when you are done, you can copy/paste it to the R console.
If you are using \href{http://www.rstudio.com/}{RStudio}, open a new `R
script', and type your model syntax (and all other R commands needed for
this session) in the source editor of RStudio. And save your script, so
you can reuse it later on.

The code piece above will produce a model syntax object, called
\texttt{myModel} that can be used later when calling a function that
actually estimates this model given a dataset. Note that formulas can be
split over multiple lines, and you can use comments (starting with the
\texttt{\#} character) and blank lines within the single quotes to
improve the readability of the model syntax.

You may split your model syntax is multiple parts. For example:

\begin{Shaded}
\begin{Highlighting}[]
\NormalTok{part1 }\OtherTok{\textless{}{-}} \StringTok{\textquotesingle{}   \# latent variable definitions }
\StringTok{               f1 =\textasciitilde{} y1 + y2 + y3 }
\StringTok{               f2 =\textasciitilde{} y4 + y5 + y6 }
\StringTok{               f3 =\textasciitilde{} y7 + y8 + y9 + y10 }
\StringTok{         \textquotesingle{}}
\NormalTok{part2 }\OtherTok{\textless{}{-}} \StringTok{\textquotesingle{}   \# fix covariance between f1 and f2 to zero}
\StringTok{               f1 \textasciitilde{}\textasciitilde{} 0*f2}
\StringTok{         \textquotesingle{}}
\end{Highlighting}
\end{Shaded}

When fitting the model, you may then simply concatenate the multiple
parts together as follows:

\begin{Shaded}
\begin{Highlighting}[]
\NormalTok{fit }\OtherTok{\textless{}{-}} \FunctionTok{cfa}\NormalTok{(}\AttributeTok{model =} \FunctionTok{c}\NormalTok{(part1, part2), }\AttributeTok{data =}\NormalTok{ myData)}
\end{Highlighting}
\end{Shaded}
