If you have no full dataset, but you do have a sample covariance matrix,
you can still fit your model. If you wish to add a mean structure, you
need to provide a mean vector too. Importantly, if only sample
statistics are provided, you must specify the number of observations
that were used to compute the sample moments. The following example
illustrates the use of a sample covariance matrix as input. First, we
read in the lower half of the covariance matrix (including the
diagonal):

\begin{Shaded}
\begin{Highlighting}[]
\NormalTok{lower }\OtherTok{\textless{}{-}} \StringTok{\textquotesingle{}}
\StringTok{ 11.834}
\StringTok{  6.947   9.364}
\StringTok{  6.819   5.091  12.532}
\StringTok{  4.783   5.028   7.495   9.986}
\StringTok{ {-}3.839  {-}3.889  {-}3.841  {-}3.625  9.610}
\StringTok{{-}21.899 {-}18.831 {-}21.748 {-}18.775 35.522 450.288 \textquotesingle{}}

\NormalTok{wheaton.cov }\OtherTok{\textless{}{-}} 
    \FunctionTok{getCov}\NormalTok{(lower, }\AttributeTok{names =} \FunctionTok{c}\NormalTok{(}\StringTok{"anomia67"}\NormalTok{, }\StringTok{"powerless67"}\NormalTok{, }
                            \StringTok{"anomia71"}\NormalTok{, }\StringTok{"powerless71"}\NormalTok{,}
                            \StringTok{"education"}\NormalTok{, }\StringTok{"sei"}\NormalTok{))}
\end{Highlighting}
\end{Shaded}

The \texttt{getCov()} function makes it easy to create a full covariance
matrix (including variable names) if you only have the lower-half
elements (perhaps pasted from a textbook or a paper). Note that the
lower-half elements are written between two single quotes. Therefore,
you have some additional flexibility. You can add comments, and blank
lines. If the numbers are separated by a comma, or a semi-colon, that is
fine too. For more information about \texttt{getCov()}, see the online
manual page.

Next, we can specify our model, estimate it, and request a summary of
the results:

\begin{Shaded}
\begin{Highlighting}[]
\CommentTok{\# classic wheaton et al. model}
\NormalTok{wheaton.model }\OtherTok{\textless{}{-}} \StringTok{\textquotesingle{}}
\StringTok{  \# latent variables}
\StringTok{    ses     =\textasciitilde{} education + sei}
\StringTok{    alien67 =\textasciitilde{} anomia67 + powerless67}
\StringTok{    alien71 =\textasciitilde{} anomia71 + powerless71}
\StringTok{  \# regressions}
\StringTok{    alien71 \textasciitilde{} alien67 + ses}
\StringTok{    alien67 \textasciitilde{} ses}
\StringTok{  \# correlated residuals}
\StringTok{    anomia67 \textasciitilde{}\textasciitilde{} anomia71}
\StringTok{    powerless67 \textasciitilde{}\textasciitilde{} powerless71}
\StringTok{\textquotesingle{}}
\NormalTok{fit }\OtherTok{\textless{}{-}} \FunctionTok{sem}\NormalTok{(wheaton.model, }
           \AttributeTok{sample.cov =}\NormalTok{ wheaton.cov, }
           \AttributeTok{sample.nobs =} \DecValTok{932}\NormalTok{)}
\FunctionTok{summary}\NormalTok{(fit, }\AttributeTok{standardized =} \ConstantTok{TRUE}\NormalTok{)}
\end{Highlighting}
\end{Shaded}

\begin{verbatim}
lavaan 0.6-11 ended normally after 84 iterations

  Estimator                                         ML
  Optimization method                           NLMINB
  Number of model parameters                        17
                                                      
  Number of observations                           932
                                                      
Model Test User Model:
                                                      
  Test statistic                                 4.735
  Degrees of freedom                                 4
  P-value (Chi-square)                           0.316

Parameter Estimates:

  Standard errors                             Standard
  Information                                 Expected
  Information saturated (h1) model          Structured

Latent Variables:
                   Estimate  Std.Err  z-value  P(>|z|)   Std.lv  Std.all
  ses =~                                                                
    education         1.000                               2.607    0.842
    sei               5.219    0.422   12.364    0.000   13.609    0.642
  alien67 =~                                                            
    anomia67          1.000                               2.663    0.774
    powerless67       0.979    0.062   15.895    0.000    2.606    0.852
  alien71 =~                                                            
    anomia71          1.000                               2.850    0.805
    powerless71       0.922    0.059   15.498    0.000    2.628    0.832

Regressions:
                   Estimate  Std.Err  z-value  P(>|z|)   Std.lv  Std.all
  alien71 ~                                                             
    alien67           0.607    0.051   11.898    0.000    0.567    0.567
    ses              -0.227    0.052   -4.334    0.000   -0.207   -0.207
  alien67 ~                                                             
    ses              -0.575    0.056  -10.195    0.000   -0.563   -0.563

Covariances:
                   Estimate  Std.Err  z-value  P(>|z|)   Std.lv  Std.all
 .anomia67 ~~                                                           
   .anomia71          1.623    0.314    5.176    0.000    1.623    0.356
 .powerless67 ~~                                                        
   .powerless71       0.339    0.261    1.298    0.194    0.339    0.121

Variances:
                   Estimate  Std.Err  z-value  P(>|z|)   Std.lv  Std.all
   .education         2.801    0.507    5.525    0.000    2.801    0.292
   .sei             264.597   18.126   14.597    0.000  264.597    0.588
   .anomia67          4.731    0.453   10.441    0.000    4.731    0.400
   .powerless67       2.563    0.403    6.359    0.000    2.563    0.274
   .anomia71          4.399    0.515    8.542    0.000    4.399    0.351
   .powerless71       3.070    0.434    7.070    0.000    3.070    0.308
    ses               6.798    0.649   10.475    0.000    1.000    1.000
   .alien67           4.841    0.467   10.359    0.000    0.683    0.683
   .alien71           4.083    0.404   10.104    0.000    0.503    0.503
\end{verbatim}

\hypertarget{the-sample.cov.rescale-argument}{%
\paragraph{\texorpdfstring{The \texttt{sample.cov.rescale}
argument}{The sample.cov.rescale argument}}\label{the-sample.cov.rescale-argument}}

If the estimator is \texttt{ML} (the default), then the sample
variance-covariance matrix will be rescaled by a factor (N-1)/N. The
reasoning is the following: the elements in a sample variance-covariance
matrix have (usually) been divided by N-1. But the (normal-based) ML
estimator would divide the elements by N. Therefore, we need to rescale.
If you don't want this to happen (for example in a simulation study),
you can provide the argument \texttt{sample.cov.rescale\ =\ FALSE}.

\hypertarget{multiple-groups}{%
\paragraph{Multiple groups}\label{multiple-groups}}

If you have multiple groups, the \texttt{sample.cov} argument must be a
list containing the sample variance-covariance matrix of each group as a
separate element in the list. If a mean structure is needed, the
\texttt{sample.mean} argument must be a list containing the sample means
of each group. Finally, the \texttt{sample.nobs} argument can be either
a list or an integer vector containing the number of observations for
each group.
